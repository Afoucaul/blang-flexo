\documentclass{article}

\usepackage{imta_core}
\usepackage{imta_extra}
\usepackage{adjustbox}

\author{Armand Foucault}
\date{Novembre 2017}
\title{Métamodélisation du langage B et \mbox{interactions} avec un prouveur B}
\subtitle{Plan de travail, rapport bibliographique}

% \imtaSetIMTStyle

\newcommand{\rawHref}[1]{\hspace{0.2em}\textcolor{imtaLightBlue}{\href{#1}{#1}}\hspace{0.2em}}
\usepackage{colortbl}
\newcolumntype{L}[1]{>{\raggedright\let\newline\\\arraybackslash\hspace{0pt}}m{#1}}
\newcolumntype{C}[1]{>{\centering\let\newline\\\arraybackslash\hspace{0pt}}m{#1}}
\newcolumntype{R}[1]{>{\raggedleft\let\newline\\\arraybackslash\hspace{0pt}}m{#1}}
\colorlet{Header}{imtaGreen}
\definecolor{Task}{RGB}{209, 231, 151}
\definecolor{Subtask}{RGB}{236, 245, 214}
% \colorlet{Header}{imtaLightBlue}
% \definecolor{Task}{RGB}{179, 242, 255}
% \definecolor{Subtask}{RGB}{230, 251, 255}


%%%%%%%%%%%%%%%%%%%%%%%%%%%%%%% 
%%%%%%%%%% BEGINNING %%%%%%%%%% 
\begin{document}

\imtaMaketitlepage

\tableofcontents

\newpage

% PRESENTATION DU PROJET

\section{Présentation du projet}

\subsection{Contexte}

Dans le cadre du projet FORMOSE, le client se reposait initialement sur la méthode B, et sa version évènementielle, Event-B.
Cependant, ces méthodes se sont avérées peu cohérentes dans leur approche avec la philosophie du projet.
Ainsi, le client souhaite développer une nouvelle méthode formelle, dérivée de la méthode B, et pilotée par la spécification de buts.
Le client ayant déjà modélisé les obligations de preuve de cette nouvelle méthode formelle, il reste à implémenter la lecture de celles-ci dans l'assistant logiciel qu'il utilise, Atelier B.

Par ailleurs, le client utilise OpenFlexo pour unifier les différentes technologies avec lesquelles il travaille.
Ainsi, OpenFlexo constitue l'interface principale de travail, à travers laquelle les systèmes sont conçus, et spécifiés en langage B.
Toutes les obligations de preuves rédigées dans OpenFlexo sont transmises à l'assistant de preuve, qui les lit, les exécute, et en retourne le résultat à OpenFlexo.
Il sera donc nécessaire de développer un \textit{technology adapter} pour Openflexo, permettant d'interfacer ce dernier avec l'implémentation de la nouvelle méthode dans l'assistant logiciel.\\

\subsection{Objectifs}

L'outil privilégié par le client est Atelier B, l'assistant officiel et propriétaire de la méthode B.
Cependant, ce logiciel étant propriétaire, il a été convenu de s'orienter vers Rodin, son pendant open-source, dont le développement de plugins est bien plus accessible.

À la lumière de ces éléments, ce projet consiste donc en :

\vspace{\baselineskip}
\begin{itemize}
    \item L'implémentation d'un plugin Rodin capable de lire les nouvelles obligations de preuve conçues par le client
    \item La réalisation d'un technology adapter connectant OpenFlexo avec Rodin
\end{itemize}

\newpage


% PLAN DE TRAVAIL

\section{Plan de travail}

Les axes principaux de ce projet sont les suivants :

\vspace{\baselineskip}
\begin{itemize}
    \item Compréhension du formalisme et de la théorie de la méthode B
    \item Manipulation et compréhension de Rodin
    \item Manipulation et compréhension d'OpenFlexo
    \item Conception d'un plugin Rodin pour la lecture des nouvelles obligations de preuve
    \item Implémentation d'un technology adapter pour l'interfaçage d'OpenFlexo avec Rodin
\end{itemize}
\vspace{\baselineskip}

Voici le plan de travail prévisionnel pour le déroulement du projet :

\vspace{\baselineskip}

\begin{adjustbox}{center}
    \def\arraystretch{1.2}
    \begin{tabular}{| L{0.55\linewidth} C{\linewidth/7} R{0.14\linewidth}  L{0.13\linewidth} |}
        \hline
        \rowcolor{Header}\multicolumn{1}{|c}{Tâche} & \multicolumn{1}{c}{Temps consacré} & \multicolumn{1}{r}{Du} & \multicolumn{1}{l|}{Au}\\ \hline \hline

        \rowcolor{Task}Compréhension de la méthode B & 2 semaines & 6 novembre & 17 novembre\\
        \rowcolor{Subtask}\qquad Lecture du cours de Marie-Laure Potet & 2 semaines & 6 novembre & 17 novembre\\ \hline


        \rowcolor{Task}Familiarisation avec Rodin & 2 semaines & 20 novembre & 1\(^{er}\) décembre\\
        \rowcolor{Subtask}\qquad Étude des exercices du Rodin Handbook & 2 semaines & 20 novembre & 1\(^{er}\) décembre\\ \hline

        \rowcolor{Task}Familiarisation avec OpenFlexo & 3 semaines & 4 décembre & 22 décembre\\
        \rowcolor{Subtask}\qquad Étude de Free Modeling Editor et de Viewpoint Modeler & 1 semaine & 4 décembre & 8 décembre\\
        \rowcolor{Subtask}\qquad Étude de Smartdocs & 1 semaine & 11 décembre & 15 décembre\\
        \rowcolor{Subtask}\qquad Étude de View Editor & 1 semaine & 18 décembre & 22 décembre\\ \hline

        \rowcolor{Task}Conception d'un plugin Rodin & 5 semaines & 8 janvier & 9 février \\
        \rowcolor{Subtask}\qquad Compréhension des nouvelles obligations de preuve & 1 semaine & 8 janvier & 12 janvier\\
        \rowcolor{Subtask}\qquad Étude du tutoriel d'Aymerick Savary pour Rodin & 1 semaine & 15 janvier & 19 janvier\\
        \rowcolor{Subtask}\qquad Développement du plugin & 3 semaines & 22 janvier & 9 février\\ \hline

        \rowcolor{Task}Implémentation d'un technology adapter OpenFlexo & 4 semaines & 12 février & 9 mars\\
        \rowcolor{Subtask}\qquad Étude de technoloy adapters existants & 1 semaine & 12 février & 16 février\\
        \rowcolor{Subtask}\qquad Développement du technology adapater & 3 semaines & 19 février & 9 mars\\ \hline \hline

        \rowcolor{Header}\multicolumn{1}{|c}{Total} & \multicolumn{1}{c}{17 semaines} & 6 novembre 17 & 9 mars 18\\ \hline

    \end{tabular}
\end{adjustbox}

\newpage


% REFERENCES BIBLIOGRAPHIQUES

\section{Références bibliographiques}

\subsection{Théorie de la méthode B}

\begin{itemize}
    \item The B Method, Marie-Laure Potet, 2011 \\ \rawHref{http://www-verimag.imag.fr/\textasciitilde{}potet/ejcp-expose.pdf}
    \item Méthode B, Wikipédia \\ \rawHref{https://fr.wikipedia.org/wiki/Méthode\_B}
\end{itemize}

\subsection{Rodin}

\begin{itemize}
    \item Guide utilisateur Rodin, methode-b.com \\ \rawHref{http://www.methode-b.com/outils/rodin/guide-utilisateur/}
    \item Wiki de la méthode Event-B et de Rodin \\ \rawHref{http://wiki.event-b.org/index.php/Main\_Page}
    \item Rodin User's Handbook \\ \rawHref{http://www3.hhu.de/stups/handbook/rodin/current/html/tutorial.html}
    \item Développer pour Rodin, Aymerick Savary \\ \rawHref{http://blog.aymericksavary.fr/?p=557}
\end{itemize}

\subsection{OpenFlexo}

\begin{itemize}
    \item Documentation officielle, openflexo.org \\ \rawHref{https://developers.openflexo.org/developers/documentation.html}
\end{itemize}

% OUTILS

\section{Outils}

\begin{itemize}
    \item \LaTeX : rédaction des documents
    \item Rodin : manipulation et compréhension de la méthode B
    \item Git : suivi du code et des documents de gestion de projet
\end{itemize}

\end{document}
%%%%%%%%%% END %%%%%%%%%% 
%%%%%%%%%%%%%%%%%%%%%%%%% 
