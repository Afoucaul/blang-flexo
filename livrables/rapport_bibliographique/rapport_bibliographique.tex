\documentclass{article}

\usepackage{imta_core}
\usepackage{imta_extra}
\usepackage{adjustbox}
\usepackage[backend=biber, sorting=none]{biblatex}
\usepackage[francais]{babel}

\usepackage{csquotes}
\addbibresource{../common/bibliography.bib}

\author{Armand Foucault}
\date{Novembre 2017}
\title{Métamodélisation du langage B et \mbox{interactions} avec un prouveur B}
\subtitle{Plan de travail, rapport bibliographique}

% \imtaSetIMTStyle

\newcommand{\rawHref}[1]{\hspace{0.2em}\textcolor{imtaLightBlue}{\href{#1}{#1}}\hspace{0.2em}}
\usepackage{colortbl}
\newcolumntype{L}[1]{>{\raggedright\let\newline\\\arraybackslash\hspace{0pt}}m{#1}}
\newcolumntype{C}[1]{>{\centering\let\newline\\\arraybackslash\hspace{0pt}}m{#1}}
\newcolumntype{R}[1]{>{\raggedleft\let\newline\\\arraybackslash\hspace{0pt}}m{#1}}
\colorlet{Header}{imtaGreen}
\definecolor{Task}{RGB}{209, 231, 151}
\definecolor{Subtask}{RGB}{236, 245, 214}
% \colorlet{Header}{imtaLightBlue}
% \definecolor{Task}{RGB}{179, 242, 255}
% \definecolor{Subtask}{RGB}{230, 251, 255}


%%%%%%%%%%%%%%%%%%%%%%%%%%%%%%% 
%%%%%%%%%% BEGINNING %%%%%%%%%% 
\begin{document}

\imtaMaketitlepage

\tableofcontents

\newpage

% PRESENTATION DU PROJET

\section{Présentation du projet}

\subsection{Contexte}

Ce projet s'inscrit dans le cadre du projet Formose, dont l'objectif est de concevoir une méthode formelle pour l'ingénierie %
des exigences, soutenue par un écosystème open-source.
Le projet Formose s'attache à créer un lien entre l'univers de l'ingénierie des exigences, lié par essence au langage naturel, %
et celui de la preuve formelle, reposant sur un formalisme mathématique strict.
Afin de connecter ces deux univers, le projet Formose fait appel à OpenFlexo, une infrastructure logicielle open-source, permettant de fédérer les modèles.
OpenFlexo permet en l'occurrence de bâtir un pont entre l'ingénierie des exigences et la preuve formelle.

Dans le cadre du projet Formose, le client se repose sur la méthode B, et sa version évènementielle, Event-B, pour la rédaction de preuves formelles.
Cependant, il manque à ces méthodes des constructions et des expressions pour traduire plus naturellement les spécifications propres au projet Formose.
Le client souhaite ainsi développer une nouvelle méthode formelle, dérivée comme Event-B de la méthode B, mais pilotée par la spécification de buts.
L'idée est donc de modifier la méthode B, dont la philosophie convient déjà au projet, afin de lui apporter l'expressivité dont le projet Formose a besoin.
Dans cette optique, le client a déjà modélisé les nouvelles obligations de preuve, et souhaite dorénavant une implémentation de cette nouvelle méthode dans l'assistant logiciel qu'il utilise.

\subsection{Objectifs}

Ce projet est vaste, et laisse une grande liberté, la nouvelle méthode formelle conçue par le client n'étant pas encore complètement spécifiée.
Ainsi, avant toute implémentation, il sera nécessaire de comprendre la méthode B, et de conceptualiser ce qu'est une preuve formelle en langage B.

Une fois cette formalisation faite, viendra l'implémentation de cette nouvelle méthode dans l'assistant logiciel de preuve.
L'outil privilégié par le client est Atelier B, l'assistant officiel de la méthode B.
Cependant, ce logiciel étant propriétaire, il a été convenu de s'orienter vers Rodin, son pendant open-source, dont le développement de plugins est bien plus accessible.

Enfin, il faudra connecter la nouvelle version de Rodin avec OpenFlexo.
À cette fin, un \textit{technology adapter} devra être conçu et implémenté dans OpenFlexo.
L'intérêt de disposer d'une telle interface est de rendre la communication entre OpenFlexo et Rodin dynamique : une modification dans le premier peut être répercutée instantanément %
dans le second, qui peut répondre après traitement au premier.

À la lumière de ces éléments, ce projet consiste donc en trois objectifs :

\vspace{\baselineskip}
\begin{itemize}
    \item La formalisation d'une preuve en langage B
    \item L'implémentation d'un plugin Rodin capable de lire les nouvelles obligations de preuve conçues par le client
    \item La réalisation d'un technology adapter connectant OpenFlexo avec Rodin
\end{itemize}

\newpage


% PLAN DE TRAVAIL

\section{Plan de travail}

Les axes principaux de ce projet sont les suivants :

\vspace{\baselineskip}
\begin{itemize}
    \item Compréhension du formalisme et de la théorie de la méthode B
    \item Manipulation et compréhension de Rodin
    \item Manipulation et compréhension d'OpenFlexo
    \item Conception d'un plugin Rodin pour la lecture des nouvelles obligations de preuve
    \item Implémentation d'un technology adapter pour l'interfaçage d'OpenFlexo avec Rodin
\end{itemize}
\vspace{\baselineskip}

Voici le plan de travail prévisionnel pour le déroulement du projet :

\vspace{\baselineskip}

\begin{adjustbox}{center}
    \def\arraystretch{1.2}
    \begin{tabular}{| L{0.55\linewidth} C{\linewidth/7} R{0.14\linewidth}  L{0.13\linewidth} |}
        \hline
        \rowcolor{Header}\multicolumn{1}{|c}{Tâche} & \multicolumn{1}{c}{Temps consacré} & \multicolumn{1}{r}{Du} & \multicolumn{1}{l|}{Au}\\ \hline \hline

        \rowcolor{Task}Compréhension de la méthode B & 2 semaines & 6 novembre & 17 novembre\\
        \rowcolor{Subtask}\qquad Lecture du cours de Marie-Laure Potet & 2 semaines & 6 novembre & 17 novembre\\ \hline

        \rowcolor{Task}Familiarisation avec Rodin & 2 semaines & 20 novembre & 1\(^{er}\) décembre\\
        \rowcolor{Subtask}\qquad Étude des exercices du Rodin Handbook & 2 semaines & 20 novembre & 1\(^{er}\) décembre\\ \hline

        \rowcolor{Task}Familiarisation avec OpenFlexo & 3 semaines & 4 décembre & 22 décembre\\
        \rowcolor{Subtask}\qquad Étude de Free Modeling Editor et de Viewpoint Modeler & 1 semaine & 4 décembre & 8 décembre\\
        \rowcolor{Subtask}\qquad Étude de Smartdocs & 1 semaine & 11 décembre & 15 décembre\\
        \rowcolor{Subtask}\qquad Étude de View Editor & 1 semaine & 18 décembre & 22 décembre\\ \hline

        \rowcolor{Task}Conception d'un plugin Rodin & 5 semaines & 8 janvier & 9 février \\
        \rowcolor{Subtask}\qquad Compréhension des nouvelles obligations de preuve & 1 semaine & 8 janvier & 12 janvier\\
        \rowcolor{Subtask}\qquad Étude du tutoriel d'Aymerick Savary pour Rodin & 1 semaine & 15 janvier & 19 janvier\\
        \rowcolor{Subtask}\qquad Développement du plugin & 3 semaines & 22 janvier & 9 février\\ \hline

        \rowcolor{Task}Implémentation d'un technology adapter OpenFlexo & 4 semaines & 12 février & 9 mars\\
        \rowcolor{Subtask}\qquad Étude de technoloy adapters existants & 1 semaine & 12 février & 16 février\\
        \rowcolor{Subtask}\qquad Développement du technology adapater & 3 semaines & 19 février & 9 mars\\ \hline \hline

        \rowcolor{Header}\multicolumn{1}{|c}{Total} & \multicolumn{1}{c}{17 semaines} & 6 novembre 17 & 9 mars 18\\ \hline

    \end{tabular}
\end{adjustbox}

\subsection{Compréhension de la méthode B}

Il s'agit de comprendre la théorie mathématique sur laquelle repose la méthode B.
Pour cela, le cours de Marie-Laure Potet \cite{mlpotet} fournit, en plus d'une introduction à Atelier B, une présentation des axiomes et des outils que le formalisme %
de la méthode B met à notre disposition.
La page Wikipédia de la méthode B \cite{wikibmethod} fournira des informations complémentaires.
Il n'est pas nécessaire de comprendre en profondeur la théorie de la méthode B, mais en saisir les bases permettra une meilleure formalisation d'une preuve en langage B.

\subsection{Familiarisation avec Rodin}

L'un des objectifs étant l'implémentation d'un plugin Rodin, il est nécessaire de connaître les rudiments du développement avec Rodin.
Pour ce faire, le guide utilisateur de Rodin \cite{rodinuserguide} offre une courte présentation permettant une prise en main des bases de l'outil.
De plus, le Rodin Handbook \cite{rodinuserhandbook} propose des exercices, qui permettront de manipuler Rodin, mais aussi d'assimiler les concepts de la méthode B en les appliquant.
Le wiki officiel de Rodin et de la méthode Event-B \cite{rodinwiki} sera également un support utile. 

\subsection{Familiarisation avec OpenFlexo}

OpenFlexo est une infrastructure composée de nombreux outils, permettant de manipuler les différents modèles de données à disposition afin de les fédérer.
La compréhension de ces outils, à l'aide des tutoriels disponibles sur le site d'OpenFlexo \cite{openflexodoc}, permettra de mieux appréhender la portée de cet écosystème.

\subsection{Conception d'un plugin Rodin}

Ayant étudié au préalable l'outil Rodin, la conception d'un plugin implémentant la nouvelle méthode formelle pourra commencer.
Il faudra tout d'abord comprendre les attentes exactes du client, ce qui sera rendu plus aisé par la connaissance des outils et la compréhention de la méthode B.
Ensuite, le tutoriel d'Aymerick Savary \cite{asavary} permettra d'appréhender les bases du développement pour Rodin.
Enfin viendra le développement réel du plugin, accompagné de tests et de la rédaction d'une documentation au fur et à mesure.

\subsection{Implémentation d'un technology adapter OpenFlexo}

Une fois le plugin Rodin implémenté et fonctionnel, un technology adapter devra être réalisé, pour permettre à OpenFlexo de communiquer avec la nouvelle méthode formelle.
Le site officiel d'OpenFlexo \cite{openflexodoc} propose déjà plusieurs technology adapters, qui seront dans un premier temps à étudier pour comprendre leur fonctionnement.
Enfin, le technology adapter réel sera développé, testé, et documenté.

\section{Outils}

Au cours de ce projet, différents outils logiciels seront utilisés.

\subsection{Git}

Git sera utilisé pour gérer le versionnage du code et des documents rédigés en \LaTeX.
Le dépôt est hébergé sur la plateforme Redmine d'IMT Atlantique, et disponible à l'adresse suivante : \href{%
    https://redmine-df.telecom-bretagne.eu/git/blang-metamodelisation}{https://redmine-df.telecom-bretagne.eu/git/blang-metamodelisation}.

\subsection{\LaTeX}

Les livrables seront rédigés en \LaTeX, afin de produire les documents de la meilleure qualité possible, et pour rendre aisé leur suivi.

\subsection{Redmine}

Le code et les livrables seront hébergés sur la plateforme Redmine de l'école, disponible à l'adresse suivante : \href{%
https://redmine-df.telecom-bretagne.eu/projects/blang-metamodelisation}{https://redmine-df.telecom-bretagne.eu/projects/blang-metamodelisation}.

\subsection{Rodin}

Rodin est l'assistant logiciel de preuve formelle B utilisé dans le cadre de ce projet.
Le téléchargement est disponible à l'adresse suivante : \href{http://www.event-b.org/install.html}{http://www.event-b.org/install.html}.


\newpage

\nocite{*}

\printbibheading[title=Références, heading=bibintoc]

\printbibliography[keyword=bmethod, heading=subbibintoc, title={Méthode B}]

\printbibliography[keyword=rodin, heading=subbibintoc, title={Rodin}]

\printbibliography[keyword=openflexo, heading=subbibintoc, title={OpenFlexo}]


\end{document}
%%%%%%%%%% END %%%%%%%%%% 
%%%%%%%%%%%%%%%%%%%%%%%%% 
