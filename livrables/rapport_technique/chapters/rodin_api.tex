\chapter{Étude de l'API Rodin}

\section{Éléments de machines Event-B}

L'API de Rodin définit, dans les packages préfixés par \javacode{org.eventb.}, les éléments relatifs à la description de systèmes %
avec la méthode Event-B.
Elle définit d'une part les protocoles de ces éléments sous la forme d'interfaces, et leur implémentation en tant que classes.
Dans cette section, nous présentons d'abord les protocoles implémentés par les éléments de la méthode Event-B, puis les classes qui leur font écho.

\subsection{Protocoles}

L'API de Rodin définit, dans le package \javacode{org.eventb.core}, les protocoles que chacun des types d'éléments %
d'une machine B doit respecter.
Comme le cœur de Rodin est écrit en Java, il est naturel que ces protocoles soient des interfaces.
Nous avons donc, pour chaque type d'élément Event-B, le contrat à respecter, représenté par une interface.
Nous retrouvons ainsi les interfaces \javacode{IAction}, \javacode{IEvent}, \javacode{IInvariant}...
Nous nous proposons de présenter, à titre d'exemple mais aussi pour approfondir notre compréhension de l'API, l'interface \javacode{IEvent}.

\subsubsection{Un exemple : \texttt{IEvent}}

L'interface \javacode{IEvent} présente principalement des accesseurs, permettant d'accéder aux sous-éléments d'un évènement Event-B.
Ces éléments sont les suivants~:

\begin{itemize}
    \item Les actions \javacode{IAction};
    \item Les gardes, implémentant \javacode{IGuard};
    \item Les variables locales, implémentant \javacode{IVariable};
    \item Les clauses de raffinement, implémentant \javacode{IRefinesEvent};
    \item Les témoins, implémentant \javacode{IWitness};
\end{itemize}

L'interface \javacode{IEvent} impose donc l'existence d'accesseurs à ces sous-éléments.
Ces accesseurs sont de deux types : nommé, et exhaustif.

Les accesseurs nommés permettent d'accéder à un sous-élément par son nom.
Ainsi, la méthode \javacode{getGuard} prend un \javacode{String} en argument, et renvoie une instance de \javacode{IGuard}, portant le nom spécifié.
Cette instance peut ne pas exister dans l'instance d'\javacode{IEvent}.
Cela signifie que la méthode \javacode{getGuard} renvoie systématiquement une garde portant le nom demandé, mais que celle-ci %
peut ne pas être attachée à l'évènement concerné.

Les accesseurs exhaustifs, quant à eux, renvoient la liste des sous-éléments de l'élément concernés.
Ainsi, la méthode \javacode{getActions} ne prend aucun argument, et renvoie un tableau de type \javacode{IActions[]}, dont nous savons que tous %
les éléments sont effectivement rattachés à l'évènement.

L'interface \javacode{IEvent} présente finalement les accesseurs suivants~:

\begin{itemize}
    \item \javacode{IAction getAction(String name)} et \javacode{IAction[] getActions()}
    \item \javacode{IGuard getGuard(String name)} et \javacode{IGuard[] getGuards()}
    \item \javacode{IParameter getParameter(String name)} et \javacode{IParameter[] getParameters()}
    \item \javacode{IRefinesEvent getRefinesClause(String name)} et \javacode{IRefinesEvent[] getRefinesClauses()}
    \item \javacode{IWitness getWitness(String name)} et \javacode{IWitness[] getWitnesses()}
\end{itemize}

Nous notons que les variables locales se retrouvent sous le nom de \javacode{Parameter}.

\subsubsection{Généralisation avec \texttt{IInternalElement}}

Les protocoles d'élément Event-B dérivent de la super-interface \javacode{IInternalElement}.
Cette interface permet entre autres de manipuler plus génériquement les sous-éléments des éléments Event-B.
Ainsi, les accesseurs nommés spécialisés tels que \javacode{getAction(String name)} ou \javacode{getInvariant(String name)} sont généralisés par %
la méthode \javacode{getInternalElement}, qui prend en paramètres le type de l'élément auquel accéder, et son nom.
Le type est défini comme une instance de \javacode{IInternalElementType<T extends IInternalElement>}, %
% ce qui permet de travailler avec des éléments de type indéterminé, en lui passant une instance de \javacode{IInternalElementType<IInternalElement>}.
sachant que les interfaces d'élément Event-B définissent chacune leur type à travers une variable statique \javacode{ELEMENT_TYPE}, instance de %
\javacode{IInternalElementType<T>} où \javacode{T} est l'interface elle-même.


\subsection{Implémentation}

L'implémentation des éléments Event-B et des interfaces du packages \javacode{org.eventb.core} est réalisée par les classes contenues dans %
le package \javacode{org.eventb.core.basis}.
Ainsi, chaque protocole défini dans le package \javacode{org.eventb.core} trouve sa réalisation, comme \javacode{IAction} implémentée par %
\javacode{Action}, ou encore \javacode{IGuard} implémentée par \javacode{Guard}.

Les implémentations de ces interfaces ne sont toutefois pas destinées à être utilisées telles quelles.
La documentation suggère en effet aux clients de l'API de ne manipuler que des instances des interfaces, ou bien à étendre les implémentations.
En reprenant l'exemple de l'élément \javacode{Event}, cela revient à ne travailler qu'avec des \javacode{IEvent}, ou alors à créer une classe fille de \javacode{Event}


\section{Éléments de projet Rodin}

L'API de Rodin fournit une surcouche aux éléments Event-B pour les manipuler en tant qu'éléments d'un projet Rodin, c'est-à-dire d'un type de projet Eclipse.
Les interfaces et classes auxquelles nous nous intéresseons sont ainsi définies dans le package \javacode{org.rodinp.core}.

\subsection{La base de données Rodin}

Nous nous proposons tout d'abord de présenter le cœur de Rodin : la base de données Rodin.
Implémentée par la classe \javacode{RodinDB}, celle-ci est chargée de conserver en mémoire les éléments de projets créés.
Toute manipulation d'élément de projet Rodin entraîne une écriture dans la base de données.
Par ailleurs, cette classe implémente un singleton, en lançant une exception de type \javacode{java.lang.Error} si son constructeur est appelée deux fois.

La classe \javacode{RodinDB} propose différentes méthodes de manipulation de projet, notamment \javacode{getWorkspace} et \javacode{getRodinProject}.
Elle est par ailleurs gérée par la classe \javacode{RodinDBManager}, qui la rend disponible à travers sa méthode \javacode{getRodinDB}.


\subsection{Protocole}

Le package \javacode{org.rodinp.core} définit tout d'abord le protocole des élément de projet Rodin, à travers l'interface \javacode{IRodinElement}.
Cette interface permet notamment d'interagir avec la base de données Rodin, avec la méthode \javacode{exists}, vérifiant si l'élément existe dans la base.
Elle permet également d'observer la hiérarchie d'éléments, à travers les méthodes \javacode{getAncestor} et \javacode{getParent}.

Par ailleurs, l'interface \javacode{IRodinElement} fournit également des méthodes de manipulation générique des éléments, telles que \javacode{getElementName} et %
\javacode{getElementType}.


\subsection{Implémentation}\label{sec:rodinProjectElementProtocol}

L'implémentation des éléments de projet Rodin et de l'interface \javacode{IRodinElement} est réalisée par la classe \javacode{RodinElement}.

En plus des principales méthodes que nous avons évoquées en section \ref{sec:rodinProjectElementProtocol}, la classe \javacode{RodinElement} fournit %
des accesseurs supplémentaires, dont \javacode{getChildren} et \javacode{getChildrenOfType}, renvoyant chacun une séquence d'éléments fils de l'élément courant.


\section{Résumé : l'exemple de la classe \texttt{Event}}\label{sec:rodinApiSummaryEvent}

Nous résumons dans cette section les fonctionnalités d'une classe implémentant un élément Event-B.
Pour rester dans la même continuité, nous nous intéressons à l'implémentation d'un évènement, à travers la classe \javacode{Event}.
Nous présentons donc les méthodes qu'elle met à notre disposition, ainsi que les classes et interfaces dont elle les tient.

La figure suivante présente l'arbre d'héritage de la classe \javacode{Event}, annoté des interfaces implémentées par chacune des classes.

\begin{figure}[H]
\centering
\begin{imtaConsole}
java.lang.Object
│
└── org.eclipse.core.runtime.PlatformObject
    │
    └── org.rodinp.core.basis.RodinElement              -> IRodinElement
        │
        └── org.rodinp.core.basis.InternalElement       -> IInternalElement, IParent,
            │                                              IElementManipulation
            └── org.eventb.core.basis.EventBElement
                │
                └── org.eventb.core.basis.Event         -> IEvent
\end{imtaConsole}
\caption{Arbre d'héritage annoté de la classe \texttt{Event}}
\label{fig:inheritanceTreeEvent}
\end{figure}

Soit une instance de la classe \javacode{Event}, que nous nommons \javacode{event}.

À travers l'interface \javacode{IEvent}, nous pouvons utiliser les méthodes \javacode{event.getActions}, \javacode{event.getGuards}, \javacode{event.getParameters}, %
\javacode{event.getRefinesClauses}, et \javacode{event.getWitnesses}, ainsi que leurs pendants nommés, pour accéder respectivement aux actions, aux gardes, aux variables locales, %
aux clauses de raffinement, et aux témoins de l'évènement \javacode{event}.

Les méthodes héritées de \javacode{EventBElement} permettent de manipuler \javacode{event} comme un élément de projet Event-B.
Nous pouvons ainsi décider qu'\javacode{event} est un théorème avec \javacode{event.setTheorem}, obtenir le commentaire qui lui est associé avec \javacode{event.getComment}, %
et obtenir son nom et son expression \textit{via} les méthodes \javacode{event.getIdentifierString} et \javacode{event.getExpressionString}.

L'interface \javacode{IInternalElement} rapproche \javacode{event} de Rodin, en lui ajoutant des méthodes agissant sur la base de données Rodin.
Ainsi, nous pouvons forcer la création d'\javacode{event} dans cette dernière avec \javacode{event.create}, et lui attribuer un fils avec \javacode{event.createChild}.
Nous pouvons également comparer \javacode{event} à un autre élément, avec \javacode{event.hasSameAttributes}, ou encore \javacode{event.hasSameChildren}.
Nous pouvons par ailleurs interagir avec l'environnement de travail, et notamment récupérer le fichier courant avec \javacode{event.getRodinFile}.
