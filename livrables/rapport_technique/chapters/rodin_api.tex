\chapter{Étude de l'API Rodin}

\section{Éléments de machines Event-B}

L'API de Rodin définit, dans les packages préfixés par \javacode{org.eventb.}, les éléments relatifs à la description de systèmes %
avec la méthode Event-B.
Elle définit d'une part les protocoles de ces éléments sous la forme d'interfaces, et leur implémentation en tant que classes.
Dans cette section, nous présentons d'abord les protocoles implémentés par les éléments de la méthode Event-B, puis les classes qui leur font écho.

\subsection{Protocoles}

L'API de Rodin définit, dans le package \javacode{org.eventb.core}, les protocoles que chacun des types d'éléments %
d'une machine B doit respecter.
Comme le cœur de Rodin est écrit en Java, il est naturel que ces protocoles soient des interfaces.
Nous avons donc, pour chaque type d'élément Event-B, le contrat à respecter, représenté par une interface.
Nous retrouvons ainsi les interfaces \javacode{IAction}, \javacode{IEvent}, \javacode{IInvariant}...
Nous nous proposons de présenter, à titre d'exemple mais aussi pour approfondir notre compréhension de l'API, l'interface \javacode{IEvent}.

\subsubsection{Un exemple : \texttt{IEvent}}

L'interface \javacode{IEvent} présente principalement des accesseurs, permettant d'accéder aux sous-éléments d'un évènement Event-B.
Ces éléments sont les suivants~:

\begin{itemize}
    \item Les actions \javacode{IAction};
    \item Les gardes, implémentant \javacode{IGuard};
    \item Les variables locales, implémentant \javacode{IVariable};
    \item Les clauses de raffinement, implémentant \javacode{IRefinesEvent};
    \item Les témoins, implémentant \javacode{IWitness};
\end{itemize}

L'interface \javacode{IEvent} impose donc l'existence d'accesseurs à ces sous-éléments.
Ces accesseurs sont de deux types : nommé, et exhaustif.

Les accesseurs nommés permettent d'accéder à un sous-élément par son nom.
Ainsi, la méthode \javacode{getGuard} prend un \javacode{String} en argument, et renvoie une instance de \javacode{IGuard}, portant le nom spécifié.
Cette instance peut ne pas exister dans l'instance d'\javacode{IEvent}.
Cela signifie que la méthode \javacode{getGuard} renvoie systématiquement une garde portant le nom demandé, mais que celle-ci %
peut ne pas être attachée à l'évènement concerné.

Les accesseurs exhaustifs, quant à eux, renvoient la liste des sous-éléments de l'élément concernés.
Ainsi, la méthode \javacode{getActions} ne prend aucun argument, et renvoie un tableau de type \javacode{IActions[]}, dont nous savons que tous %
les éléments sont effectivement rattachés à l'évènement.

L'interface \javacode{IEvent} présente finalement les accesseurs suivants~:

\begin{itemize}
    \item \javacode{IAction getAction(String name)} et \javacode{IAction[] getActions()}
    \item \javacode{IGuard getGuard(String name)} et \javacode{IGuard[] getGuards()}
    \item \javacode{IParameter getParameter(String name)} et \javacode{IParameter[] getParameters()}
    \item \javacode{IRefinesEvent getRefinesClause(String name)} et \javacode{IRefinesEvent[] getRefinesClauses()}
    \item \javacode{IWitness getWitness(String name)} et \javacode{IWitness[] getWitnesses()}
\end{itemize}

Nous notons que les variables locales se retrouvent sous le nom de \javacode{Parameter}.

\subsubsection{Généralisation}

Les protocoles d'élément Event-B dérivent de la super-interface \javacode{IInternalElement}.
Cette interface permet entre autres de manipuler plus génériquement les sous-éléments des éléments Event-B.
Ainsi, les accesseurs nommés spécialisés tels que \javacode{getAction(String name)} ou \javacode{getInvariant(String name)} sont généralisés par %
la méthode \javacode{getInternalElement}, qui prend en paramètres le type de l'élément auquel accéder, et son nom.
Le type est défini comme une instance de \javacode{IInternalElementType<T extends IInternalElement>}, %
% ce qui permet de travailler avec des éléments de type indéterminé, en lui passant une instance de \javacode{IInternalElementType<IInternalElement>}.
sachant que les interfaces d'élément Event-B définissent chacune leur type à travers une variable statique \javacode{ELEMENT_TYPE}, instance de %
\javacode{IInternalElementType<T>} où \javacode{T} est l'interface elle-même.


\subsection{Implémentation}

% XXX


\section{Éléments de projet Rodin}

L'API de Rodin fournit une surcouche aux éléments Event-B pour les manipuler en tant qu'éléments d'un projet Rodin, c'est-à-dire d'un projet Eclipse.
Les interfaces ainsi définies sont contenues dans le package \javacode{org.rodinp.core}.


\section{Résumé : l'exemple de la classe \texttt{Event}}

Nous présentons sur la figure suivante l'arbre d'héritage de la classe \javacode{Event}, annoté des interfaces implémentées par chacune des classes.

\begin{imtaConsole}
java.lang.Object
└── org.eclipse.core.runtime.PlatformObject
    └── org.rodinp.core.basis.RodinElement              -> IRodinElement
        └── org.rodinp.core.basis.InternalElement       -> IInternalElement
            └── org.eventb.core.basis.EventBElement
                └── org.eventb.core.basis.Event         -> IEvent
\end{imtaConsole}
