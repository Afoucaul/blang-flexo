\chapter{Développement d'un Technology Adapter Rodin}

\section{Étude de la documentation de l'API Rodin}

\section{Architecture du Technology Adapter}

\section{Réalisation du Technology Adapter}

\subsection{Création d'un projet Rodin}

Dépendances : org.eclipse.core.resources

Le wiki Rodin fournit une solution permettant de créer dynamiquement et programmatiquement des projets, \textit{via} l'API.
Nous réutilisons ce code, et l'intégrons au plugin de test sous la forme d'une méthode \imtaInlinecode{java}{createRodinProject}.



Dépendances

\begin{itemize}
    \item org.eclipse.core.resources
    \item org.eclipse.core.runtime
    \item org.eclipse.equinox.common
    \item org.eclipse.equinox.registry
    \item org.eventb.core
    \item org.rodinp.core
    \item org.eclipse.osgi
\end{itemize}
