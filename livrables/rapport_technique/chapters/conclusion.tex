\chapter*{Conclusion et perspectives}
\label{sec:conclusion}
\addcontentsline{toc}{chapter}{\nameref{sec:conclusion}}


\section*{Validation de la preuve de concept}
\label{sec:validation}
\addcontentsline{toc}{section}{\nameref{sec:validation}}

Nous sommes parvenus à concevoir et à développer une abstraction minimaliste et opérationnelle de l'API de Rodin.
Nous avons par la suite réussi à faire communiquer Rodin avec une application extérieure, afin de faire exécuter des instructions à l'éditeur.
Nous avons pu démontrer le concept de cette communication, en jouant un scénario présentant diverses opérations constructives de manipulation de projet Rodin.
Bien que la communication avec Rodin ne soit pas encore implémentée du côté d'OpenFlexo, le principe de l'interaction a pu être validé.

Notons par ailleurs que le principe de la communication par TCP entre Rodin et OpenFlexo peut être étendu à n'importe quelle autre application Eclipse RCP.
Ainsi, la base de communication que nous avons conçue pourra être réemployée pour intégrer une autre application fondée sur Eclipse à OpenFlexo.


\section*{Perspectives d'amélioration}
\label{sec:perspectives}
\addcontentsline{toc}{section}{\nameref{sec:perspectives}}

Pour conclure cette preuve de concept, il reste à concevoir et à implémenter la communication du côté d'OpenFlexo.
Cela pourra se présenter sous la forme d'un \textit{Technology Adapter}, une bibliothèque de classes Java visant à créer un lien entre OpenFlexo et un modèle.
Il faudra prendre garde à l'unification des commandes, et éventuellement réfléchir à un protocole de communication, afin que Rodin et OpenFlexo se comprennent.

Par ailleurs, certains points de cette preuve de concept restent améliorables.
En premier lieu, l'environnement artificiel que nous utilisons pour manipuler les projets dans le cadre du plugin devrait être remplacé par un lien avec la base de données Rodin.
De plus, l'exécution des instructions s'effectue sur le même thread que la réception de messages.
Il faudra étudier le flux d'instructions potentiel venant d'OpenFlexo, pour décider s'il est nécessaire de séparer la réception et l'exécution comme il est traditionnellement fait.
Enfin, le langage d'instructions est encore pauvre, et il faudra l'enrichir afin notamment de permettre la modification d'éléments existants, et la prise en charge d'éléments créés hors %
du plugin.
