\chapter{Conception de la surcouche Rodin}

Nous souhaitons abstraire les fonctionnalités de gestion de projet de l'API Rodin, afin de fournir une surcouche simple d'utilisation.
Faisant écho aux scénarios de validation, nous présentons d'abord les fonctions abstraites que nous voulons pouvoir appeler sur cette surcouche.


\section{Aperçu de la surcouche}

Nous nous proposons de créer une classe \imtaInlinecode{java}{RodinApiHandle}, qui va proposer les différentes fonctions de la surcouche.
Cette classe présentera l'interface suivante :

% \begin{table}[H]
%     \centering
%     \begin{tabular}{| l | l | l |}
%         \hline
%         \multicolumn{1}{|c|}{Modificateurs} & \multicolumn{1}{|c|}{Signature} & \multicolumn{1}{|c|}{Description}\\
%         \hline\hline
%         \texttt{public static} & \texttt{IRodinProject createRodinProject(final String name)} & Création d'un projet Rodin\\
%         \hline
%         \texttt{public static} & \texttt{IRodinProject createEventbMachine(IRodinProject, final String name)} & Création d'une machine Event-B dans un projet\\
%         \hline
%     \end{tabular}
%     \caption{Interface externe de la surcouche Rodin}
%     \label{table:rodinApiHandle}
% \end{table}

\vspace{\baselineskip}
\begin{labeling}{public static}
    \newcommand{\javacode}[1]{\texttt{#1}}
    \setlength{\itemsep}{1.5em}

    \item [\javacode{public static}] \javacode{IRodinProject createRodinProject(final String name)}
        \begin{itemize}[label={}]
            \item Création d'un projet Rodin avec le nom donné.
        \end{itemize}

    \item [\javacode{public static}] \javacode{IMachineRoot createEventbMachine(IRodinProject project, final String name)}
        \begin{itemize}[label={}]
            \item Création d'une machine Event-B dans le projet donné et avec le nom donné.
        \end{itemize}

\end{labeling}
