\chapter*{Présentation du projet}
\label{sec:introduction}
\addcontentsline{toc}{chapter}{\nameref{sec:introduction}}

\section*{Contexte}
\label{sec:contexte}
\addcontentsline{toc}{section}{\nameref{sec:contexte}}

Ce projet s'inscrit dans le cadre du projet Formose, dont l'objectif est de concevoir une méthode formelle pour l'ingénierie %
des exigences, soutenue par un écosystème open-source.
Le projet Formose s'attache à créer un lien entre l'univers de l'ingénierie des exigences, lié par essence au langage naturel, %
et celui de la preuve formelle, reposant sur un formalisme mathématique strict.
Afin de connecter ces deux univers, le projet Formose fait appel à OpenFlexo, une infrastructure logicielle open-source, permettant de fédérer les modèles.
OpenFlexo permet en l'occurrence de bâtir un pont entre l'ingénierie des exigences et la preuve formelle.

Dans le cadre du projet Formose, le client se repose sur la méthode B, et sa version évènementielle, Event-B, pour la rédaction de preuves formelles.
Cependant, le projet Formose se place dans le réferentiel de l'ingénierie des exigences, et il manque à ces méthodes des constructions et des expressions %
pour traduire plus naturellement les spécifications liées à ce réferentiel.
Le client souhaite ainsi développer une nouvelle méthode formelle, dérivée comme Event-B de la méthode B, mais pilotée par la spécification de buts, afin de mieux correspondre %
à l'ingénierie des exigences.
L'idée est donc de modifier la méthode B, dont la philosophie convient déjà au projet, et de lui apporter l'expressivité dont le projet Formose a besoin.
Dans cette optique, le client a déjà modélisé les nouvelles obligations de preuve, et souhaite dorénavant une implémentation de cette nouvelle méthode dans l'assistant logiciel qu'il utilise.


\section*{Choix de la réalisation et objectifs}
\label{sec:objectifs}
\addcontentsline{toc}{section}{\nameref{sec:objectifs}}

Ce projet comporte une dimension très exploratoire, et laisse une grande liberté dans le choix de la réalisation.
Par ailleurs, le client a finalisé l'implémentation du métamodèle de la méthode Event-B dans OpenFlexo.
Il incombe donc à notre projet de déterminer d'une part l'outil de méthode B que nous souhaitons intégrer dans OpenFlexo, et d'autre part l'approche à adopter pour procéder à cette intégration.
La contribution finale de ce projet consiste en une preuve de concept de cette intégration.

Le choix de l'environnement de méthode B avec lequel travailler est relativement immédiat.
En effet, bien que l'outil privilégié par le client soit Atelier B, l'assistant officiel de la méthode B, celui-ci est propriétaire, et la majeure partie de son code source est fermée.
Dans l'optique de réaliser une preuve de concept, nous avons décidé de travailler avec Rodin, en ayant pour perspective future l'intégration d'Atelier B à OpenFlexo.

L'approche à adopter pour réaliser cette intégration est difficile à déterminer en amont du projet.
La faisabilité des solutions que nous pouvons envisager \textit{a priori} ne peut se dessiner qu'à la suite d'une étude de l'API de Rodin.
Nous pouvons cependant d'ores et déjà considérer deux familles de solutions~:

\begin{itemize}
    \item Isolation du cœur de Rodin et intégration dans OpenFlexo
    \item Communication entre une instance de Rodin et OpenFlexo
\end{itemize}

Dans un cas comme dans l'autre, nous aurons besoin de développer un \textit{Technology Adapter} OpenFlexo afin de réaliser effectivement l'intégration.
\newline

Les objectifs de ce projet, auquel le présent document s'attache à répondre, sont donc~:

\begin{itemize}
    \item L'asbtraction de l'API de Rodin pour préparer son intégration dans OpenFlexo~;
    \item La réalisation d'un \textit{Technology Adapter} pour concrétiser cette intégration.
\end{itemize}
