\documentclass{article}

%%%%%%%%%% PACKAGES %%%%%%%%%%

% Document geometry management
\usepackage[a4paper, margin=2cm, top=3cm]{geometry}

% Images management
\usepackage{graphicx}

% Floats management
\usepackage{float}

% Output font management
\usepackage[T1]{fontenc}

% Document encoding management
\usepackage[utf8]{inputenc}

% Include external pdf documents
% Used for generating the cover
\usepackage{pdfpages}

% Fancy headers and footers
\usepackage{fancyhdr}

% Pictures drawing
\usepackage{tikz}

% Metadata access
\usepackage{titling}

% Arbitrary font size
\usepackage{anyfontsize}

% Fancy frames and boxes
\usepackage{mdframed}

% Section styling
\usepackage{sectsty}

% Fancy icons
\usepackage{fontawesome}



%%%%%%%%%% COLORS %%%%%%%%%%

% The three colors of IMTA: green, light blue, and dark blue
\definecolor{imtaGreen}{RGB}{164, 210, 51}
\definecolor{imtaLightBlue}{RGB}{0, 184, 222}
\definecolor{imtaDarkBlue}{RGB}{12, 35, 64}
\definecolor{imtaGray}{RGB}{87, 87, 87}



%%%%%%%%%% GENERAL SETTINGS %%%%%%%%%%

\raggedbottom

\sectionfont{\sc}
\subsectionfont{\sc}



%%%%%%%%%% PACKAGES SETTINGS %%%%%%%%%%

% graphicx
\setkeys{Gin}{width=\linewidth}


% fancyhdr
\pagestyle{fancy}
\fancyhead{}
\fancyfoot{}
\fancyhead[L]{\nouppercase\thetitle{} \meetingnoValue}
\fancyhead[R]{\thedate}

\fancypagestyle{imtaFirstpage}{%
    \fancyhf{}
    \renewcommand{\headrulewidth}{0pt}
}



%%%%%%%%%% COMMANDS %%%%%%%%%%

% \subtitle
% Command for the cover's subtitle
\newcommand{\subtitleValue}{}
\newcommand{\subtitle}[1]{%
    \renewcommand{\subtitleValue}{#1}
}

% \meetingno
% Command for the title
\newcommand{\meetingnoValue}{}
\newcommand{\meetingno}[1]{%
    \renewcommand{\meetingnoValue}{#1}
}


% \imtammMaketitle
% Output the title on one line
\newcommand{\imtammMaketitle}{%
    \begin{center}
        \textsc{\textbf{\LARGE\thetitle{} \meetingnoValue}}\\
        \vspace{0.5\baselineskip}
        {\Large\subtitleValue}
    \end{center}
}


% \imtaMaketitlepage
% Output the IMTA title page
\newcommand{\imtaMaketitlepage}{%
    \thispagestyle{imtaFirstpage}
    \pagenumbering{gobble}
    \includepdf[pagecommand={%
        \begin{tikzpicture}[remember picture, overlay]
        \node[anchor=west, align=left] at (-0.5, -10.5) {%
            \large\fontfamily{phv}\selectfont\thedate \\
            \large\fontfamily{phv}\selectfont\theauthor \vspace{1.5cm}\\
            \textcolor{imtaGreen}{\fontsize{25}{40}\fontseries{b}\fontfamily{phv}\selectfont\thetitle} \vspace{.5cm}\\
            \textcolor{imtaGreen}{\Large\fontfamily{phv}\selectfont\subtitleValue}
        };
        \end{tikzpicture}
    }]{titlepage}
    \newpage
    \pagenumbering{arabic}
    \setcounter{page}{2}
}


% \imtaSetIMTStyle
% Set a styling that conforms to the official report template
\newcommand{\imtaSetIMTStyle}{%
    % Set global font to Helvetica
    \usepackage{helvet}
    \renewcommand{\familydefault}{\sfdefault}

    % Set heading style
    \sectionfont{\bf\LARGE\color{imtaGreen}}
    \subsectionfont{\bf\Large\color{imtaGray}}
    \subsubsectionfont{\bf\large\color{imtaGray}}
    \paragraphfont{\color{imtaGray}}
    \subparagraphfont{\color{imtaGray}}

    % Set header and footer style
    \pagestyle{fancy}
    \fancyhead{}
    \fancyfoot{}
    \fancyhead[L]{\nouppercase\subtitleValue}
    \fancyhead[R]{\thedate}
    \fancypagestyle{imtaFirstpage}{%
        \fancyhf{}
        \renewcommand{\headrulewidth}{0pt}
    }
}



%%%%%%%%%% ENVIRONMENTS %%%%%%%%%%

% imtammActorList
% Typeset a list of actors
\newenvironment{imtammActorList}{%
    \begin{itemize}
    \renewcommand\labelitemi{\faUser}
}{%
    \end{itemize}
}


\meetingno{1}
\author{Armand Foucault}
\date{27 octobre 2017}
\title{Compte rendu de réunion}
\subtitle{Rencontre client, établissement des besoins}

\newcommand{\newactor}[2]{\expandafter\newcommand\csname actor#1\endcsname{#2}}
\newactor{jcbach}{Dr Jean-Christophe Bach, maître de conférences - Encadrant du projet}
\newactor{abeugnard}{Dr Antoine Beugnard, enseignant chercheur}
\newactor{asavary}{Dr Aymerick Savary, chercheur en science informatique - Client}
\newactor{afoucaul}{Armand Foucault, élève ingénieur - Réalisateur du projet}


\begin{document}

\imtammMaketitle

\section*{Participants}

\begin{imtammActorList}
\item \actorjcbach
\item \actorabeugnard
\item \actorasavary
\item \actorafoucaul
\end{imtammActorList}


\section*{Discussion}

\subsection*{Présentation du contexte}

Ce projet s'inscrit dans le cadre du projet FORMOSE\footnote{%
http://www.agence-nationale-recherche.fr/projet-anr/?tx\_lwmsuivibilan\_pi2\%5BCODE\%5D=ANR-14-CE28-0009}, %
projet de recherche industrielle visant à la production d'une méthode formelle d'ingénierie des exigences.
Parmi les outils au coeur de ce projet, se trouve notamment \textbf{OpenFlexo}\footnote{%
https://www.openflexo.org/}, une méthode de création d'outils et d'unification de modèles.
Le client travaille avec OpenFlexo pour représenter ses preuves formelles reposant sur la \textbf{méthode B}.
L'outil qu'il développe à travers OpenFlexo interagit avec un prouveur B, qui génère et vérifie les preuves conçues avec cet outil.

\subsection*{Expression des attentes}

Dans le cadre du projet FORMOSE, la méthode B n'était pas parfaitement appropriée.
Le client s'est ainsi tourné vers la version événementielle de B, appelée Event B.
Aujourd'hui, le client ne tire pas pleine satisfaction de la méthode Event B.
En particulier, les obligations de preuve proposées par Event B ne correspondent pas à la philosophie du projet Formose.
Par conséquent, le client souhaite modifier certains aspects de la méthode Event B, afin de l'orienter vers la modélisation de buts (\textit{goal modeling}).

Pour mettre en oeuvre la méthode B, le client utilise le logiciel \textbf{Atelier B}.
Cependant, il s'agit d'un logiciel propriétaire, dont la modification pourrait s'avérer compliquée, et requerrait trop de temps.
Il existe un autre outil de déploiement de la méthode B, nommé \textbf{Rodin}.
Ce dernier est un projet open-source, conçu pour accueillir des plugins développés en Java.
L'intégration d'une nouvelle version de la méthode B à cet outil est donc a priori accessible.

Finalement, nous nous attacherons à la conception d'un plugin Java pour Rodin.
L'objectif est donc avant tout de réaliser une preuve de concept de la nouvelle version de la méthode B souhaitée par le client.
Par ailleurs, le client réalisant principalement ses modèles depuis OpenFlexo, il est souhaité que la nouvelle version de Rodin s'interface avec ce dernier.
À cette fin, il sera nécessaire de réaliser un \textbf{Technology Adapter}, qui permettra à OpenFlexo d'interagir avec Rodin.


\section*{À venir}

Pour la mi-novembre 2017, rédaction d'un document comprenant :

\begin{itemize}
    \item Les objectifs du projet
    \item Le plan de travail
    \item Les références bibliographiques
\end{itemize}

\noindent D'ici décembre 2017 :

\begin{itemize}
\item Comprendre le contexte du projet et les exigences du client
\item Décider de comment le Technology Adapter sera réalisé
\item Expliciter la forme, les fonctionnalités... du produit final
\end{itemize}

\end{document}
