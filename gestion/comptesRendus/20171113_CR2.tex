\documentclass{article}

%%%%%%%%%% PACKAGES %%%%%%%%%%

% Document geometry management
\usepackage[a4paper, margin=2cm, top=3cm]{geometry}

% Images management
\usepackage{graphicx}

% Floats management
\usepackage{float}

% Output font management
\usepackage[T1]{fontenc}

% Document encoding management
\usepackage[utf8]{inputenc}

% Include external pdf documents
% Used for generating the cover
\usepackage{pdfpages}

% Fancy headers and footers
\usepackage{fancyhdr}

% Pictures drawing
\usepackage{tikz}

% Metadata access
\usepackage{titling}

% Arbitrary font size
\usepackage{anyfontsize}

% Fancy frames and boxes
\usepackage{mdframed}

% Section styling
\usepackage{sectsty}

% Fancy icons
\usepackage{fontawesome}



%%%%%%%%%% COLORS %%%%%%%%%%

% The three colors of IMTA: green, light blue, and dark blue
\definecolor{imtaGreen}{RGB}{164, 210, 51}
\definecolor{imtaLightBlue}{RGB}{0, 184, 222}
\definecolor{imtaDarkBlue}{RGB}{12, 35, 64}
\definecolor{imtaGray}{RGB}{87, 87, 87}



%%%%%%%%%% GENERAL SETTINGS %%%%%%%%%%

\raggedbottom

\sectionfont{\sc}
\subsectionfont{\sc}



%%%%%%%%%% PACKAGES SETTINGS %%%%%%%%%%

% graphicx
\setkeys{Gin}{width=\linewidth}


% fancyhdr
\pagestyle{fancy}
\fancyhead{}
\fancyfoot{}
\fancyhead[L]{\nouppercase\thetitle{} \meetingnoValue}
\fancyhead[R]{\thedate}

\fancypagestyle{imtaFirstpage}{%
    \fancyhf{}
    \renewcommand{\headrulewidth}{0pt}
}



%%%%%%%%%% COMMANDS %%%%%%%%%%

% \subtitle
% Command for the cover's subtitle
\newcommand{\subtitleValue}{}
\newcommand{\subtitle}[1]{%
    \renewcommand{\subtitleValue}{#1}
}

% \meetingno
% Command for the title
\newcommand{\meetingnoValue}{}
\newcommand{\meetingno}[1]{%
    \renewcommand{\meetingnoValue}{#1}
}


% \imtammMaketitle
% Output the title on one line
\newcommand{\imtammMaketitle}{%
    \begin{center}
        \textsc{\textbf{\LARGE\thetitle{} \meetingnoValue}}\\
        \vspace{0.5\baselineskip}
        {\Large\subtitleValue}
    \end{center}
}


% \imtaMaketitlepage
% Output the IMTA title page
\newcommand{\imtaMaketitlepage}{%
    \thispagestyle{imtaFirstpage}
    \pagenumbering{gobble}
    \includepdf[pagecommand={%
        \begin{tikzpicture}[remember picture, overlay]
        \node[anchor=west, align=left] at (-0.5, -10.5) {%
            \large\fontfamily{phv}\selectfont\thedate \\
            \large\fontfamily{phv}\selectfont\theauthor \vspace{1.5cm}\\
            \textcolor{imtaGreen}{\fontsize{25}{40}\fontseries{b}\fontfamily{phv}\selectfont\thetitle} \vspace{.5cm}\\
            \textcolor{imtaGreen}{\Large\fontfamily{phv}\selectfont\subtitleValue}
        };
        \end{tikzpicture}
    }]{titlepage}
    \newpage
    \pagenumbering{arabic}
    \setcounter{page}{2}
}


% \imtaSetIMTStyle
% Set a styling that conforms to the official report template
\newcommand{\imtaSetIMTStyle}{%
    % Set global font to Helvetica
    \usepackage{helvet}
    \renewcommand{\familydefault}{\sfdefault}

    % Set heading style
    \sectionfont{\bf\LARGE\color{imtaGreen}}
    \subsectionfont{\bf\Large\color{imtaGray}}
    \subsubsectionfont{\bf\large\color{imtaGray}}
    \paragraphfont{\color{imtaGray}}
    \subparagraphfont{\color{imtaGray}}

    % Set header and footer style
    \pagestyle{fancy}
    \fancyhead{}
    \fancyfoot{}
    \fancyhead[L]{\nouppercase\subtitleValue}
    \fancyhead[R]{\thedate}
    \fancypagestyle{imtaFirstpage}{%
        \fancyhf{}
        \renewcommand{\headrulewidth}{0pt}
    }
}



%%%%%%%%%% ENVIRONMENTS %%%%%%%%%%

% imtammActorList
% Typeset a list of actors
\newenvironment{imtammActorList}{%
    \begin{itemize}
    \renewcommand\labelitemi{\faUser}
}{%
    \end{itemize}
}

\usepackage{hologo}

\meetingno{2}
\author{Armand Foucault}
\date{13 novembre 2017}
\title{Compte rendu de réunion}
\subtitle{Retour sur le rapport court}

\newcommand{\newactor}[2]{\expandafter\newcommand\csname actor#1\endcsname{#2}}
\newactor{jcbach}{Dr Jean-Christophe Bach, maître de conférences - Encadrant du projet}
\newactor{abeugnard}{Dr Antoine Beugnard, enseignant chercheur}
\newactor{asavary}{Dr Aymerick Savary, chercheur en science informatique - Client}
\newactor{afoucaul}{Armand Foucault, élève ingénieur - Réalisateur du projet}


\begin{document}

\imtammMaketitle

\section*{Participants}

\begin{imtammActorList}
\item \actorjcbach
\item \actorafoucaul
\end{imtammActorList}


\section*{Discussion}

\subsection*{Présentation du contexte}

Le premier livrable est à livrer pour le 17 novembre 2017.
Il s'agit d'un rapport court, comprenant :

\begin{itemize}
    \item La contextualisation du projet ;
    \item Une reformulation du problème ;
    \item Une bibliographie ;
    \item Une analyse du problème ;
    \item Un plan de travail prévisionnel.
\end{itemize}

Une première version de ce rapport court ayant été produite, M. Jean-Christophe Bach fait part de ses remarques pour la version suivante.

\subsection*{Éléments à étoffer}

\subsubsection*{Contextualisation du projet}

Le contexte du projet FORMOSE doit être explicité.
En particulier, celui-ci lie l'ingénierie des exigences, souvent en langage naturel, et l'univers de la preuve formelle.
De ce fait, il est naturel que le projet FORMOSE fasse appel à OpenFlexo, dont l'objet est de fédérer les univers conceptuels.

L'essence d'OpenFlexo est également à préciser.
Il s'agit d'une infrastructure permettant de fédérer des modèles, jouant le rôle de glue entre des univers de formalismes bien distincts.
C'est donc un outil de choix pour créer le lien entre l'ingénierie des exigences et la preuve formelle.

\subsubsection*{Objectifs du projet}

Il faut ajouter aux objectifs du projet la compréhension et la formalisation de la preuve formelle B.
En effet, ce projet est très libre, et les différents univers doivent être clairement définis.
Ainsi, il faut mettre en avant la phase de recherche, qui précède toute implémentation, et qui permettra de conceptualiser ce qu'est une preuve B.

\subsubsection*{Plan de travail prévisionnel}

Il serait bon d'expliciter les tâches, en ajoutant une sous-section pour chacune d'entre elles.
Ainsi, les dépendances entre les tâches pourraient être mises en évidence.
Se posent de plus les questions de l'intégration, des tests, et de la documentation.
Il faut montrer l'existence de ces questions dans le détail des tâches, et expliquer ce qui est prévu.
Enfin, il serait judicieux de préciser que certaines tâches pourraient être amenées à changer, voire à disparaître, au cours de l'avancée du projet.

\subsubsection*{Bibliographie}

La bibliographie devrait être réalisée avec le package \hologo{BibTeX} pour \LaTeX.

\section*{À venir}

Pour le 15 novembre, rédaction de la nouvelle version du rapport court.

\end{document}
