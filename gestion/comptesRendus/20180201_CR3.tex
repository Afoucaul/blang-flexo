\documentclass{article}

%%%%%%%%%% PACKAGES %%%%%%%%%%

% Document geometry management
\usepackage[a4paper, margin=2cm, top=3cm]{geometry}

% Images management
\usepackage{graphicx}

% Floats management
\usepackage{float}

% Output font management
\usepackage[T1]{fontenc}

% Document encoding management
\usepackage[utf8]{inputenc}

% Include external pdf documents
% Used for generating the cover
\usepackage{pdfpages}

% Fancy headers and footers
\usepackage{fancyhdr}

% Pictures drawing
\usepackage{tikz}

% Metadata access
\usepackage{titling}

% Arbitrary font size
\usepackage{anyfontsize}

% Fancy frames and boxes
\usepackage{mdframed}

% Section styling
\usepackage{sectsty}

% Fancy icons
\usepackage{fontawesome}



%%%%%%%%%% COLORS %%%%%%%%%%

% The three colors of IMTA: green, light blue, and dark blue
\definecolor{imtaGreen}{RGB}{164, 210, 51}
\definecolor{imtaLightBlue}{RGB}{0, 184, 222}
\definecolor{imtaDarkBlue}{RGB}{12, 35, 64}
\definecolor{imtaGray}{RGB}{87, 87, 87}



%%%%%%%%%% GENERAL SETTINGS %%%%%%%%%%

\raggedbottom

\sectionfont{\sc}
\subsectionfont{\sc}



%%%%%%%%%% PACKAGES SETTINGS %%%%%%%%%%

% graphicx
\setkeys{Gin}{width=\linewidth}


% fancyhdr
\pagestyle{fancy}
\fancyhead{}
\fancyfoot{}
\fancyhead[L]{\nouppercase\thetitle{} \meetingnoValue}
\fancyhead[R]{\thedate}

\fancypagestyle{imtaFirstpage}{%
    \fancyhf{}
    \renewcommand{\headrulewidth}{0pt}
}



%%%%%%%%%% COMMANDS %%%%%%%%%%

% \subtitle
% Command for the cover's subtitle
\newcommand{\subtitleValue}{}
\newcommand{\subtitle}[1]{%
    \renewcommand{\subtitleValue}{#1}
}

% \meetingno
% Command for the title
\newcommand{\meetingnoValue}{}
\newcommand{\meetingno}[1]{%
    \renewcommand{\meetingnoValue}{#1}
}


% \imtammMaketitle
% Output the title on one line
\newcommand{\imtammMaketitle}{%
    \begin{center}
        \textsc{\textbf{\LARGE\thetitle{} \meetingnoValue}}\\
        \vspace{0.5\baselineskip}
        {\Large\subtitleValue}
    \end{center}
}


% \imtaMaketitlepage
% Output the IMTA title page
\newcommand{\imtaMaketitlepage}{%
    \thispagestyle{imtaFirstpage}
    \pagenumbering{gobble}
    \includepdf[pagecommand={%
        \begin{tikzpicture}[remember picture, overlay]
        \node[anchor=west, align=left] at (-0.5, -10.5) {%
            \large\fontfamily{phv}\selectfont\thedate \\
            \large\fontfamily{phv}\selectfont\theauthor \vspace{1.5cm}\\
            \textcolor{imtaGreen}{\fontsize{25}{40}\fontseries{b}\fontfamily{phv}\selectfont\thetitle} \vspace{.5cm}\\
            \textcolor{imtaGreen}{\Large\fontfamily{phv}\selectfont\subtitleValue}
        };
        \end{tikzpicture}
    }]{titlepage}
    \newpage
    \pagenumbering{arabic}
    \setcounter{page}{2}
}


% \imtaSetIMTStyle
% Set a styling that conforms to the official report template
\newcommand{\imtaSetIMTStyle}{%
    % Set global font to Helvetica
    \usepackage{helvet}
    \renewcommand{\familydefault}{\sfdefault}

    % Set heading style
    \sectionfont{\bf\LARGE\color{imtaGreen}}
    \subsectionfont{\bf\Large\color{imtaGray}}
    \subsubsectionfont{\bf\large\color{imtaGray}}
    \paragraphfont{\color{imtaGray}}
    \subparagraphfont{\color{imtaGray}}

    % Set header and footer style
    \pagestyle{fancy}
    \fancyhead{}
    \fancyfoot{}
    \fancyhead[L]{\nouppercase\subtitleValue}
    \fancyhead[R]{\thedate}
    \fancypagestyle{imtaFirstpage}{%
        \fancyhf{}
        \renewcommand{\headrulewidth}{0pt}
    }
}



%%%%%%%%%% ENVIRONMENTS %%%%%%%%%%

% imtammActorList
% Typeset a list of actors
\newenvironment{imtammActorList}{%
    \begin{itemize}
    \renewcommand\labelitemi{\faUser}
}{%
    \end{itemize}
}

\usepackage{hologo}

\meetingno{3}
\author{Armand Foucault}
\date{1er février 2018}
\title{Compte rendu de réunion}
\subtitle{Redéfinition des objectifs}

\newcommand{\newactor}[2]{\expandafter\newcommand\csname actor#1\endcsname{#2}}
\newactor{jcbach}{Dr Jean-Christophe Bach, maître de conférences - Encadrant du projet}
\newactor{abeugnard}{Dr Antoine Beugnard, enseignant chercheur}
\newactor{asavary}{Dr Aymerick Savary, chercheur en science informatique - Client}
\newactor{afoucaul}{Armand Foucault, élève ingénieur - Réalisateur du projet}


\begin{document}

\imtammMaketitle

\section*{Participants}

\begin{imtammActorList}
\item \actorasavary
\item \actorjcbach
\item \actorafoucaul
\end{imtammActorList}


\section*{Présentation du contexte}
Suite à la présentation du rapport court du 18 décembre 2017, il ressort que les objectifs du projet sont à revoir.
Par ailleurs, des clarifications de la part de M. Aymerick Savary sont attendues, concernant la métamodélisation de la méthode Event-B qu'il a réalisée dans le cadre du projet FORMOSE.


\section*{Discussion}

\subsection*{Clarifications quant à la métamodélisation d'Event-B}
M. Aymerick Savary explique que l'extension de la méthode Event-B, qu'il nomme \textbf{Goal-B}, conçue dans le cadre du projet FORMOSE ajoute au langage associé des constructions complexes.
Toutefois, cette extension s'exprime en réalité par une réinterprétation des concepts existant dans la méthode Event-B.
Afin de pouvoir utliser la méthode Goal-B depuis OpenFlexo, M. Aymerick Savary a donc implémenté un métamodèle du langage Event-B, ainsi qu'un métamodèle du langage Goal-B, reposant sur ce dernier.

Le métamodèle Event-B ayant été complètement réalisé, la partie "Métamodélisation du langage B" du projet ne fait plus partie des objectifs.

\subsection*{Définition du Technology Adapater}
Le Technology Adapter souhaité dans le cadre de ce projet ne se préoccupe plus des problématiques de l'extension Goal-B, puisque celle-ci est conçue comme une réinterprétation de la méthode Event-B.
L'objectif est donc d'implémenter une communication bidirectionnelle, permettant de manipuler dans OpenFlexo les concepts de la méthode Event-B, en s'appuyant sur le moteur de Rodin.

Les fonctionnalités de ce Technology Adapter sont définies à travers un ensemble de scénarios, qui permettront sa validation.
Ces scénarios couvrent les deux sens de la communication OpenFlexo \(\leftrightarrow\) Rodin.

Les scénarios de communication d'OpenFlexo vers Rodin mettront en œuvre la création d'éléments de Rodin (modèles, machines, invariants...) sous OpenFlexo, et la répercussion %
de cette création dans le modèle Rodin.

Les scénarios de communication de Rodin vers OpenFlexo consisteront en la génération et le calcul d'\textbf{obligations de preuve} par Rodin, et la transmission de celles-ci vers OpenFlexo.


\section*{Livrables}
À l'issue de cette réunion, nous proposons, en tant que livrables finaux du projet, les documents et produits suivants :

\begin{itemize}
    \item Un Technology Adapter, établissant une communication entre OpenFlexo et Rodin ;
    \item Une compilation de scénarios de validation du Technology Adapter, accompagnés de leur diagramme de séquence ;
    \item Un rapport technique, retraçant le développement du Technology Adapter, rédigé en \LaTeX.
\end{itemize}


\section*{À venir}
Pour le 15 février, production d'un poster présentant le contexte et l'objectif du projet pour la journée portes ouvertes.


\end{document}
